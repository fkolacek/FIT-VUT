\documentclass[a4paper, 11pt, twocolumn, czech]{article}
\usepackage{times}
\usepackage[czech]{babel}
\usepackage[left=1.5cm, text={18cm, 25cm}, top=2.5cm]{geometry}
\usepackage[utf8]{inputenc}
\usepackage[IL2]{fontenc}
\usepackage{amsmath}
\usepackage{amsthm}
\usepackage{amsfonts}
\usepackage{mdwlist}

\theoremstyle{definition}
\newtheorem{definice}{Definice}[section]
\theoremstyle{definition}
\newtheorem{algoritmus}[definice]{Algoritmus}
\theoremstyle{definition}
\newtheorem{veta}{Věta}
\theoremstyle{definition}
\newtheorem{dukaz}{Důkaz}

\begin{document}
\begin{titlepage}
\begin{center}
\huge
\textsc{Fakulta informačních technologií}\\
[-0.4mm]
\textsc{Vysoké učení technické v~Brně}\\
\vspace{\stretch{0.382}}
\LARGE
Typografie a publikování -- 2. projekt
\\[-1.2mm]
Sazba dokumentů a matematických výrazů\\
\vspace{\stretch{0.618}}
\Large
\the\year\hfill František Koláček
\end{center}
\end{titlepage}

\section*{Úvod}
V této úloze si vyzkoušíme sazbu titulní strany, matema\-tických vzorců, prostředí a~dalších textových struktur obvyklých pro technicky zaměřené texty (například rovnice (\ref{rovnice1}) nebo definice \ref{definice1} na straně \pageref{definice1}).

Na titulní straně je využito sázení nadpisu podle op\-tického středu s~využitím zlatého řezu. Tento postup byl probírán na~přednášce.

\section{Matematický text}
Nejprve se podíváme na sázení matematických symbolů a~výrazů v~plynulém textu. Pro množinu $V$ označuje $\mathrm{card}(V)$ kardinalitu $V$. Pro množinu $V$ reprezentuje $V^*$ volný monoid generovaný množinou $V$ s~operací konka\-tenace. Prvek identity ve volném monoidu $V^*$ značíme symbolem $\varepsilon$. Nechť $V^+ = V^* - \{ \varepsilon \}$. Algebraicky je tedy $V^+$ volná pologrupa generovaná množinou $V$ s~operací konkatenace. Konečnou neprázdnou množinu $V$ nazvěme $abeceda$. Pro $w \in V^*$ označuje $|w|$ délku řetězce $w$. Pro $W \subseteq V$ označuje $\mathrm{occur}(w, W)$ počet výskytů symbolů z~$W$~v~řetězci $w$ a $\mathrm{sym}(w, i)$ určuje $i$-tý symbol řetězce~$w$; například $\mathrm{sym}(abcd, 3) = c$.

Nyní zkusíme sazbu definici a vět s využitím balíku \texttt{amsthm}.

\begin{definice}
\label{definice1}
\emph{Bezkontextová gramatika} je čtveřice $G = (V,T,P,S)$, kde $V$ je totální abeceda, $T \subseteq V$ je abeceda terminálů, $S \in (V - T)$ je startující symbol a $P$ je konečná množina pravidel tvaru $q\!: A \rightarrow \alpha$, kde $A \in (V - T), \alpha \in V^*$ a $q$ je návěští tohoto pravidla. Nechť $N = V - T$ značí abecedu neterminálů. Pokud $q\!:$ $A \rightarrow \alpha \in P, \gamma, \delta \in V^*$, $G$ provádí derivační krok z $\gamma A \delta$ do $\gamma \alpha \delta$ podle pravidla $q\!: A \rightarrow \alpha$, symbolicky píšeme $\gamma A \delta$ $\Rightarrow \gamma \alpha \delta$ $[q\!: A \rightarrow \alpha]$ nebo zjednodušeně $\gamma A \delta \Rightarrow \gamma \alpha \delta$. Standardním způsobem definujeme $\Rightarrow ^n$, kde $n \geq 0$. Dále definujeme tranzitivní uzávěr $\Rightarrow ^+$ a~tranzitivně-reflexivní uzávěr $\Rightarrow ^*$.
\end{definice}

Algoritmus můžeme uvádět podobně jako definice tex\-tově, nebo použít pseudokódu vysázeného ve vhodném prostředí (například \texttt{algorithm2e}).

\begin{algoritmus}
\emph{Algoritmus pro ověření bezkontextovosti gramatiky. Mějme gramatiku $G = (N,T,P,S).$}
\begin{enumerate}
\item \label{krok1} \emph{Pro každé pravidlo $p \in P$ proveď test, zda $p$ na levé straně obsahuje právě jeden symbol z $N$.}
\item \label{krok2} \emph{Pokud všechna pravidla splňují podmínku z~kro\-ku~1, tak je gramatika $G$ bezkontextová.}
\end{enumerate}
\end{algoritmus}

\begin{definice}
\emph{Jazyk} definovaný gramatikou $G$ definujeme jako $L(G) = \{w \in T^*\,|\,S \Rightarrow ^* w$\}.
\end{definice}

\subsection{Podsekce obsahující větu}

\begin{definice}
Nechť $L$ je libovolný jazyk. $L$ je \emph{bezkontextový jazyk}, když a jen když $L = L(G)$, kde $G$ je libovolná bezkontextová gramatika.
\end{definice}

\begin{definice}
Množinu $\mathcal{L}_{CF}=\{L|L $ je bez\-kon\-textový jazyk$\}$ nazýváme \emph{třídou bezkontextových jazyků}.
\end{definice}

\begin{veta}
\emph{Nechť $L_{abc}=\{a^nb^nc^n|n\geq 0\}$. Platí, že $L_{abc} \notin \mathcal{L}_{CF}$.}
\end{veta}

\begin{proof}
Důkaz se provede pomocí Pumping lemma pro bezkontextové jazyky, kdy ukážeme, že není možné, aby platilo, což bude implikovat pravdivost věty 1.
\end{proof}

\section{Rovnice a odkazy}

Složitější matematické formulace sázíme mimo plynulý text. Lze umístit několik výrazů na jeden řádek, ale pak je třeba tyto vhodně oddělit,\,například příkazem \verb%\quad%.

$$\sqrt[x^2]{y^3_0}\quad \mathbb {N}=\{0,1,2,\ldots\} \quad x^{y^y} \neq x^{yy} \quad z_{i_j} \neq z_{ij}$$

V rovnici (\ref{rovnice1}) jsou využity tři typy závorek s~různou explicitně definovanou velikostí.

\begin{equation}
\bigg\{\Big[\big(a+b\big)*c\Big]^d+1\bigg\}\quad=\quad x \label{rovnice1}
\end{equation}
$$\lim_{x\rightarrow\infty}\frac{\sin^2 x + \cos^2 x}{4}\quad=\quad y$$

V této větě vidíme, jak vypadá implicitní vysázení limity $\mathrm{lim}_{n \rightarrow \infty}f(n)$ v normálním odstavci textu. Podobně je~to i s dalšími symboly jako $\sum_1^n$ či $\bigcup_{A \in \mathcal{B}}$. V případě vzorce $\textstyle\lim\limits_{x \to 0} \frac{\sin x}{x}=1$ jsme si vynutili méně úspornou sazbu příkazem \verb%\limits%.

\begin{eqnarray}
\displaystyle\int \limits^b_a f(x)\, \mathrm{d}x &=&-\displaystyle \int^a_b f(x)\, \mathrm{d}x \label{rovnice2}\\
\Big(\sqrt[5]{x^4}\Big)' =  \Big(x^\frac{4}{5}\Big)' &= & \frac{4}{5}x^{-\frac{1}{5}} = \frac{4}{5\sqrt[5]{x}} \label{rovnice3}\\
\overline{\overline{A \vee B}} &=& \overline{\overline{A} \wedge \overline{B}} \label{rovnice4}
\end{eqnarray}

\section{Matice}

Pro sázení matic se velmi často používá prostředí \texttt{array} a závorky (\verb%\left%, \verb%\right%).

$$\left( \begin{array}{cc}
a+b & b-a \\
\widehat{\epsilon + \omega} & \hat{\pi} \\
\vec{a} &  \overleftrightarrow{AC} \\
0 & \beta \\
\end{array} \right) $$ 

$$ A~= \left\| \begin{array}{cccc}
a_{11} & a_{12} & \cdots & a_{1n} \\
a_{21} & a_{22} & \cdots & a_{2n} \\
\vdots & \vdots & \ddots & \vdots \\
a_{m1} & a_{m2} & \cdots & a_{mn} \end{array} \right\|$$ 

$$\left| \begin{array}{cc}
t & u \\
v & w \\
\end{array} \right|
= tw-uv $$

Prostředí \texttt{array} lze úspěšně využít i jinde.

$${n\choose k} = \left\{ \begin{array}{ll}
\frac{\displaystyle n!}{\displaystyle k!(n-k)!} & \quad \mbox{pro $0 \leq k~\leq n$ }\\
0 & \quad \mbox{pro $k<0$ nebo $k>n$ }\end{array} \right.$$

\section{Závěrem}

V případě, že budete potřebovat vyjádřit matematickou konstrukci nebo symbol a nebude se Vám dařit jej nalézt v samotném \LaTeX u, doporučuji prostudovat možnosti balíku maker \AmS-\LaTeX. Analogická poučka platí obecně pro jakoukoli matematickou konstrukci v \TeX u.

\end{document}

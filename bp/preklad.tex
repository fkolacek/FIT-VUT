
\begin{center}
\textbf{\huge Bachelor's Thesis Assignment}

\Large Unofficial Translation
\end{center}


\subsection*{Student}
František Koláček

\subsection*{Specialisation}
Information Technology

\subsection*{Topic}
\textbf{Automated Web Application Vulnerability Detection}

\subsection*{Category}
Web


\subsection*{Instructions}
\begin{enumerate}
	\item Study the state of the art in web application security scanning and vulnerability detection. Study, evaluate, and compare detection capabilities and features of existing open-source web application scanners, including Revok.
	\item Based on the analysis of the existing scanners and Revok, propose several new features of Revok. The features will address missing detection capabilities in Revok, for common classes of web vulnerabilities.
	\item Propose the most effective approach to the detection of each addressed class of web vulnerabilities.
	\item Implement the proposed features of Revok and evaluate the results in comparison with selected scanners applied on available web applications for penetration testers and on real open source web applications. Any found vulnerabilities shall be disclosed responsibly.
	\item Document the project results and possible further enhancements in a technical report.
\end{enumerate}

\subsection*{Supervisor}
\textbf{Marek Rychlý, RNDr., Ph.D.}, UIFS FIT VUT



